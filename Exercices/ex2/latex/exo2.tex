
\exercice{Static and dynamic types}

Given the classes \texttt{A}, \texttt{B}, \texttt{C} and \texttt{Start}, where
\texttt{Start} is the main class of the system:
\lstset{linewidth=0.85\textwidth, xleftmargin=.075\textwidth,
basicstyle=\footnotesize, keywordstyle=\bfseries, numbers=left,
numberstyle=\tiny, stepnumber=1, numbersep=7pt, language=java,  captionpos=b,
frame=shadowbox, rulesepcolor=\color{black}, mathescape=true,
commentstyle=\tiny, breakindent=10pt, breaklines, tabsize=2, fontadjust=true,
backgroundcolor=\color[gray]{0.93}}
\lstinputlisting[caption={Source code of \texttt{A}, \texttt{B},
\texttt{C} and
\texttt{Start}}]{\compilationpath/Exercices/ex2/latex/resources/code/listing.txt
}


What text is written to the console if the application is launched? Explain
your answer in terms of \squote{static type} and \squote{dynamic type} (or
\squote{runtime type}).
