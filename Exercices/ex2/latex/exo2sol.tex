\solution{Static and dynamic types}
\paragraph{\sc{Definitions}}
\begin{itemize}
 \item \textbf{Static Type:} the static type of a variable references its type
declaration at compile time
\item \textbf{Dynamic Type:} the dynamic type of a variable is the real type of
the object the varialbe is referencing.
\end{itemize}

\paragraph{\sc{Example}}
\begin{itemize}
 \item A variable \texttt{Dog} (static) can point to an object of type
\texttt{Dalmation} (dynamic).
\item \emph{Warning:} The inverse is not possible
\end{itemize}

\paragraph{\sc{Answer}}
\begin{itemize}
 \item At compilation
 \begin{itemize}
  \item The static type of \texttt{f1} is \texttt{A}.
  \item The static type of \texttt{f2} is \texttt{B}.
  \item The static type of \texttt{f3} is \texttt{C}.
 \end{itemize}
 \item At runtime
 \begin{itemize}
  \item on line 8: the dynamic type of \texttt{f1} is \texttt{A}.
  \item on line 9: the dynamic type of \texttt{f2} is \texttt{B}.
  \item on line 10: the dynamic type of \texttt{f3} is \texttt{C}.
  \item on line 11: the dynamic type of \texttt{f1} is still \texttt{A}
$\Rightarrow$ shown text is \texttt{A}.
  \item on line 12: the dynamic type of \texttt{f1} becomes \texttt{B}.
  \item on line 13: the dynamic type of \texttt{f1} is thus \texttt{B}
$\Rightarrow$ shown test is \texttt{B}.
  \item on line 14: the dynamic type of \texttt{f1} becomes \texttt{C}.
  \item on line 15: the dynamic type of \texttt{f1} is thus \texttt{C}
$\Rightarrow$ shown text is \texttt{C}.
  \item on line 16: the dynamic type of \texttt{f2} is still \texttt{B}
$\Rightarrow$ shown text is \texttt{A}.
  \item on line 17: the dynamic type of \texttt{f3} is still \texttt{C}
$\Rightarrow$ shown test is \texttt{C}.
 \end{itemize}
 \item The shown text is \texttt{A B C A C}.
\end{itemize}
